\newpage
\vspace{3mm}

\subsubsection{Μελέτη των N-bit Predictors για σταθερό πλήθος bits Hardware}
Στο σημείο αυτό επαναλαμβάνεται η μελέτη της απόδοσης των N-bit Predictors για
διαφορετικές τιμές του N = 1, 2, 2b, 4, με τη διαφοριά ότι θα ελέγξουμε την
επίδοση σε συνδυασμούς που αντιστοιχούν σε σταθερό hardware overhead 32Κ. \\

\noindent Ακολουθούν τα διαγράμματα που προέκυψαν και ο σχετικός σχολιασμός
τους:
\vspace{1em}

   \begin{minipage}{\textwidth}
      \begin{center}
         \fbox{\textlatin{\textbf{\textit{403-gcc}}}}\\
         \vspace{3mm}
         \includegraphics[width=0.65\textwidth, frame]{./graphs/4-2ii/403-gcc.png}
         \vspace{6mm}
      \end{center}
   \end{minipage}

   \begin{minipage}{\textwidth}
      \begin{center}
         \fbox{\textlatin{\textbf{\textit{429-mcf}}}}\\
         \vspace{3mm}
         \includegraphics[width=0.65\textwidth, frame]{./graphs/4-2ii/429-mcf.png}
         \vspace{6mm}
      \end{center}
   \end{minipage}

   \begin{minipage}{\textwidth}
      \begin{center}
         \fbox{\textlatin{\textbf{\textit{434-zeusmp}}}}\\
         \vspace{3mm}
         \includegraphics[width=0.65\textwidth, frame]{./graphs/4-2ii/434-zeusmp.png}
         \vspace{6mm}
      \end{center}
   \end{minipage}

   \begin{minipage}{\textwidth}
      \begin{center}
         \fbox{\textlatin{\textbf{\textit{436-cactusADM}}}}\\
         \vspace{3mm}
         \includegraphics[width=0.65\textwidth, frame]{./graphs/4-2ii/436-cactusADM.png}
         \vspace{6mm}
      \end{center}
   \end{minipage}

   \begin{minipage}{\textwidth}
      \begin{center}
         \fbox{\textlatin{\textbf{\textit{445-gobmk}}}}\\
         \vspace{3mm}
         \includegraphics[width=0.65\textwidth, frame]{./graphs/4-2ii/445-gobmk.png}
         \vspace{6mm}
      \end{center}
   \end{minipage}

   \begin{minipage}{\textwidth}
      \begin{center}
         \fbox{\textlatin{\textbf{\textit{450-soplex}}}}\\
         \vspace{3mm}
         \includegraphics[width=0.65\textwidth, frame]{./graphs/4-2ii/450-soplex.png}
         \vspace{6mm}
      \end{center}
   \end{minipage}

   \begin{minipage}{\textwidth}
      \begin{center}
         \fbox{\textlatin{\textbf{\textit{456-hmmer}}}}\\
         \vspace{3mm}
         \includegraphics[width=0.65\textwidth, frame]{./graphs/4-2ii/456-hmmer.png}
         \vspace{6mm}
      \end{center}
   \end{minipage}

   \begin{minipage}{\textwidth}
      \begin{center}
         \fbox{\textlatin{\textbf{\textit{458-sjeng}}}}\\
         \vspace{3mm}
         \includegraphics[width=0.65\textwidth, frame]{./graphs/4-2ii/458-sjeng.png}
         \vspace{6mm}
      \end{center}
   \end{minipage}

   \begin{minipage}{\textwidth}
      \begin{center}
         \fbox{\textlatin{\textbf{\textit{459-GemsFDTD}}}}\\
         \vspace{3mm}
         \includegraphics[width=0.65\textwidth, frame]{./graphs/4-2ii/459-GemsFDTD.png}
         \vspace{6mm}
      \end{center}
   \end{minipage}

   \begin{minipage}{\textwidth}
      \begin{center}
         \fbox{\textlatin{\textbf{\textit{471-omnetpp}}}}\\
         \vspace{3mm}
         \includegraphics[width=0.65\textwidth, frame]{./graphs/4-2ii/471-omnetpp.png}
         \vspace{6mm}
      \end{center}
   \end{minipage}

   \begin{minipage}{\textwidth}
      \begin{center}
         \fbox{\textlatin{\textbf{\textit{473-astar}}}}\\
         \vspace{3mm}
         \includegraphics[width=0.65\textwidth, frame]{./graphs/4-2ii/473-astar.png}
         \vspace{6mm}
      \end{center}
   \end{minipage}

   \begin{minipage}{\textwidth}
      \begin{center}
         \fbox{\textlatin{\textbf{\textit{483-xalancbmk}}}}\\
         \vspace{3mm}
         \includegraphics[width=0.65\textwidth, frame]{./graphs/4-2ii/483-xalancbmk.png}
         \vspace{6mm}
      \end{center}
   \end{minipage}

   \begin{minipage}{\textwidth}
      \begin{center}
         \fbox{\textlatin{\textbf{\textit{Geometric Average}}}}\\
         \vspace{3mm}
         \includegraphics[width=0.65\textwidth, frame]{./graphs/4-2ii/mean.png}
         \vspace{6mm}
      \end{center}
   \end{minipage}

\paragraph{Συμπεράσματα-Σχόλια}
   Παρατηρούμε πως και στους νέους συνδυασμούς για σταθερό υλικό, η καμπύλες 
   μοιάζονυ με του προγηγούμενου ερωτήματος και τα συμπεράσματα είναι ανάλογα.

    Από της μορφές των καμπυλών στα παραπάνω διαγράμματα παρατηρούμε πως 11 στα
    12 benchmarks παρουσιάζουν βελτίωση καθώς το πλήθος των bits του predictor
    αυξάνει, δηλάδή η μετρική dMPKI φθίνει καθώς τα bits αυξάνονται, παρά την
    μείωση του πλήθος των predictors. Η μόνη διαφορετική ως προς την μορφή
    καμπύλη αντιστοιχεί στο μετρόπρόγραμμα 434.zeusmp για το οποίο το μικρότερο
    MPKI αντιστοιχεί σε 2-bit predictor.

   Από το διάγραμμα των γεωμετρικών μέσων μπορούμε να αποκτήσουμε μία συνολικότερη εποπτεία, 
   και να καταλήξουμε πως και σε αυτή την περίπτωση
    \textbf{καλύτερη επιλογή είναι ο 4-bit Predictor με 8Κ entries}.

