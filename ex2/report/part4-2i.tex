\newpage
\subsection{Μελέτη των N-bit Predictors}
\vspace{3mm}

\subsubsection{Μελέτη για σταθερό αριθμό BHT Entries}
Στο σημείο αυτό μελετάται η απόδοση των N-bit Predictors για διαφορετικές τιμές
του N = 1, 2, 3, 4, ενώ τα entries διατηρούνται σταθερά και ίσα με 16Κ.Τα N-bit
υλοποιούν ένα saturating up-down counter. Επιπλέον, υλοποιείται ένας predictor
με το κάτωθι FSM:

\begin{center}
   \vspace{3mm}
      \includegraphics[width=0.65\textwidth, frame]{./imgs/fsm.png}
   \vspace{6mm}
\end{center}


\vspace{1em}    
Η σύγκριση των predictors γίνεται με βάση τα direction Mispredictions Per
Thousand Instructions (direction MPKI). Ακολουθούν τα διαγράμματα που
προέκυψαν και ο σχετικός σχολιασμός τους:

   \begin{minipage}{\textwidth}
      \begin{center}
         \fbox{\textlatin{\textbf{\textit{403-gcc}}}}\\
         \vspace{3mm}
         \includegraphics[width=0.65\textwidth, frame]{./graphs/4-2i/403-gcc.png}
         \vspace{6mm}
      \end{center}
   \end{minipage}

   \begin{minipage}{\textwidth}
      \begin{center}
         \fbox{\textlatin{\textbf{\textit{429-mcf}}}}\\
         \vspace{3mm}
         \includegraphics[width=0.65\textwidth, frame]{./graphs/4-2i/429-mcf.png}
         \vspace{6mm}
      \end{center}
   \end{minipage}

   \begin{minipage}{\textwidth}
      \begin{center}
         \fbox{\textlatin{\textbf{\textit{434-zeusmp}}}}\\
         \vspace{3mm}
         \includegraphics[width=0.65\textwidth, frame]{./graphs/4-2i/434-zeusmp.png}
         \vspace{6mm}
      \end{center}
   \end{minipage}

   \begin{minipage}{\textwidth}
      \begin{center}
         \fbox{\textlatin{\textbf{\textit{436-cactusADM}}}}\\
         \vspace{3mm}
         \includegraphics[width=0.65\textwidth, frame]{./graphs/4-2i/436-cactusADM.png}
         \vspace{6mm}
      \end{center}
   \end{minipage}

   \begin{minipage}{\textwidth}
      \begin{center}
         \fbox{\textlatin{\textbf{\textit{445-gobmk}}}}\\
         \vspace{3mm}
         \includegraphics[width=0.65\textwidth, frame]{./graphs/4-2i/445-gobmk.png}
         \vspace{6mm}
      \end{center}
   \end{minipage}

   \begin{minipage}{\textwidth}
      \begin{center}
         \fbox{\textlatin{\textbf{\textit{450-soplex}}}}\\
         \vspace{3mm}
         \includegraphics[width=0.65\textwidth, frame]{./graphs/4-2i/450-soplex.png}
         \vspace{6mm}
      \end{center}
   \end{minipage}

   \begin{minipage}{\textwidth}
      \begin{center}
         \fbox{\textlatin{\textbf{\textit{456-hmmer}}}}\\
         \vspace{3mm}
         \includegraphics[width=0.65\textwidth, frame]{./graphs/4-2i/456-hmmer.png}
         \vspace{6mm}
      \end{center}
   \end{minipage}

   \begin{minipage}{\textwidth}
      \begin{center}
         \fbox{\textlatin{\textbf{\textit{458-sjeng}}}}\\
         \vspace{3mm}
         \includegraphics[width=0.65\textwidth, frame]{./graphs/4-2i/458-sjeng.png}
         \vspace{6mm}
      \end{center}
   \end{minipage}

   \begin{minipage}{\textwidth}
      \begin{center}
         \fbox{\textlatin{\textbf{\textit{459-GemsFDTD}}}}\\
         \vspace{3mm}
         \includegraphics[width=0.65\textwidth, frame]{./graphs/4-2i/459-GemsFDTD.png}
         \vspace{6mm}
      \end{center}
   \end{minipage}

   \begin{minipage}{\textwidth}
      \begin{center}
         \fbox{\textlatin{\textbf{\textit{471-omnetpp}}}}\\
         \vspace{3mm}
         \includegraphics[width=0.65\textwidth, frame]{./graphs/4-2i/471-omnetpp.png}
         \vspace{6mm}
      \end{center}
   \end{minipage}

   \begin{minipage}{\textwidth}
      \begin{center}
         \fbox{\textlatin{\textbf{\textit{473-astar}}}}\\
         \vspace{3mm}
         \includegraphics[width=0.65\textwidth, frame]{./graphs/4-2i/473-astar.png}
         \vspace{6mm}
      \end{center}
   \end{minipage}

   \begin{minipage}{\textwidth}
      \begin{center}
         \fbox{\textlatin{\textbf{\textit{483-xalancbmk}}}}\\
         \vspace{3mm}
         \includegraphics[width=0.65\textwidth, frame]{./graphs/4-2i/483-xalancbmk.png}
         \vspace{6mm}
      \end{center}
   \end{minipage}

   \begin{minipage}{\textwidth}
      \begin{center}
         \fbox{\textlatin{\textbf{\textit{Geometric Average}}}}\\
         \vspace{3mm}
         \includegraphics[width=0.65\textwidth, frame]{./graphs/4-2i/mean.png}
         \vspace{6mm}
      \end{center}
   \end{minipage}
\vspace{3em}
\paragraph{Συμπεράσματα - Σχόλια}
    Από της μορφές των καμπυλών στα παραπάνω διαγράμματα παρατηρούμε πως 11 στα
    12 benchmarks παρουσιάζουν βελτίωση καθώς το πλήθος των bits του predictor
    αυξάνει, δηλάδή η μετρική dMPKI φθίνει καθώς τα bits αυξάνονται από 1 σε 2
    και από 2 σε 3. Η μετάβαση από 3bit σε 4bit δεν επιφέρει πάντα βελτίωση. Η
    μόνη διαφορετική ως προς την μορφή καμπύλη αντιστοιχεί στο μετρόπρόγραμμα
    434.zeusmp για το οποίο το μικρότερο MPKI αντιστοιχεί σε 2-bit predictor.
    Ωστόσο πρέπει να επισημάνουμε πως για το εν λόγω μετροπρόγραμμα η τιμή του
    MPKI είναι ήδη αρκετά χαμηλή και η διαφοροποίηση της επίδοσης που επέρχεται
    με τη χρήση διαφορετικών Nbit-Predictors είναι αρκετά μικρή (εύρος 1.2 εώς
    2.0 Misses Per KILOInstructions), άρα δεν μας επηρεάζει και πολύ.
    
    Αναοφρικά με το FSM που υλοποιήσαμε, αποδίδει καλύτερα μονάχα σε σχέση με
    τον 1bit Predictor. O αντίστοιχος 2bit Predictor που υλοποιείται ως
    saturating up-down counter αποδίδει πάντα καλύτερα σε σχέση με τον 2-bit FSM
    Predictor.
    
    Μπορούμε επίσης να αποκτήσουμε μία πιο συνολική εικόνα για το σύνολο των
    benchmarks από το διάγραμμα γεωμετρικών μέσων τιμών, όπου εύκολα
    επιβεβαιώνουμε τα προηγούμενα συμπεράσματα.  Επιπλέον, με βάση το διάγραμμα
    των μέσων, φαίνεται πως η \textbf{καλύτερη επιλογή είναι ο 16K-3-bit Predictor}.

